\chapter{Introduction}\label{chapter:introduction}

The goal of this thesis is to introduce and evaluate different ways to integrate
prior information into our data mining procedure.

The thesis starts with a chapter, which explains the needed preliminary knowledge.
It starts with a short introduction to the sparse grid discretization method and
then covers supervised learning in general, and with sparse grids in particular.

In the following chapters we discuss different ways of integrating prior information
into this learning method.
We make the following contributions:
\begin{itemize}
\item In \cref{cha:regularization} we evaluate different regularization
  methods that help us to impose smoothness constrains on our model and allows us
  to handle noise in the data.
  We begin with a discussion of regularization theory, and then with selected
  methods and their implementation.
  The chapter also contains a discussion of an alternative solver for regularized linear
  systems, that is able to handle the newly added methods.
\item In \cref{cha:grid-gen} we evaluate different grid construction methods.
  In contrast to the chapter before, we impose our prior information even before
  the inference process starts --- we modify the grid construction algorithm.
  We first discuss a generalized form of sparse grids, that allows us to modify
  the granularity of the generated grid.
  Secondly we introduce interaction-term-aware sparse grids that allow us to construct grids with a different granularity depending on the dimensions and their interaction with each other.
  The methods discussed in that chapter allow us to tackle very-high dimensional problems, that are impossible or inefficient for regular sparse grids.
\end{itemize}

Finally, there is an appendix that covers the used datasets and our preprocessing steps.
%%% Local Variables:
%%% mode: latex
%%% TeX-master: "../main"
%%% End:
