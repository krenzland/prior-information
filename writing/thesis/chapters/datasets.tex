\chapter{Datasets}\label{cha:datasets}
We use different datasets in this thesis.
In this chapter, we look at all used datasets and document the preprocessing
steps.

Each feature of each dataset was scaled to the range \([0,1]\) with the so called min-max
scaler
\begin{equation*}
  \operatorname{scale}(\bm{x}) = \frac{\bm{x} - \min({\bm{x}})}{\max{(\bm{x}) - \min(\bm{x})}}.
\end{equation*}

After that, we perform a Box--Cox transformation for a parameter \(\lambda\):
\begin{equation*}
  \operatorname{Box--Cox}(x_i) =
  \begin{cases}
    \ln(x_i) & \text{for } \lambda = 0, \\
    x_i^\lambda - 1 & \text{otherwise}.
  \end{cases}
\end{equation*}
We estimate the best choice for \(\lambda\) using the optimization procedure
contained in the statistics module of Scipy.
Note that we need to shift the data before applying this transformation by a
small positive number, because \(ln(0)\) is undefined.
The used parameters for each dataset can be found in each section.

\todo{Insert correct values into dataset tables.}
\todo{Use tabular numbers instead of old-style figures here!}
We use the following datasets:

\begin{table}[h]
\begin{tabular}[c]{lrr}
  \toprule \multicolumn{1}{r}{\textbf{Name}}
& \multicolumn{1}{r}{\textbf{No.~of predictors}}
& \multicolumn{1}{r}{\textbf{No.~of instances}}
\\\midrule
  Friedman1 & 10 & 10000 \\
  Concrete & 9 & 1030 \\
  Power Plant & 4 & 9568 \\
  Friedman9 & 4  & 200 \\
  Friedman9 & 4  & 200
\\\bottomrule
\end{tabular}
\caption[List of Datasets]{Datasets used for this thesis.
  Included is the number of predictors(i.e.~no.~of dimensions) and the number of provided training examples.}
\end{table}

The datasets that are taken from the UCI Machine Learning Repository~\cite{datasets-uci} are marked
with a `*'.

\section{Friedman}\label{sec:friedman}
The Friedman1 dataset is an artifical ten dimensional dataset, which was first
discussed by Friedman in~\cite{datasets-friedman}.
It uses ten input variables that are uniformally distributed in \([0,1]\).
The target variable is computed using
\begin{equation*}
 y = 10 \sin(\pi x_1 x_2) + 20(x_3 - 0.5)^2 + 10x_4 + 5x_5 + \varepsilon,
\end{equation*}
with \(\varepsilon \sim \mathcal{N}(0,1)\) serving as noise.
Only the first five variables are useful for prediction, the only inherent
interaction term is \(x_1 x_2\).
This makes this dataset useful for the study of automatical feature selection
--- using regularization techniques such as the lasso or the group lasso ---
and for interaction-terms aware grid generation.

The distribution of the predictors is convenient, we do not need to normalize
them.

\section{Concrete}
The concrete dataset models the compressive strength of concrete using various
predictors.
It was first discussed in~\cite{datasets-concrete}.

\section{Power Plant}
The combined cycle power plant dataset was first discussed in~\cite{datasets-powerplant}.
%%% Local Variables:
%%% mode: latex
%%% TeX-master: "../main"
%%% End:
