% Created 2016-04-03 So 16:09
\documentclass[11pt]{scrartcl}
\usepackage[utf8]{inputenc}
\usepackage[T1]{fontenc}
\usepackage{lmodern}
\usepackage{microtype}
\usepackage{fixltx2e}
\usepackage{graphicx}
\usepackage{grffile}
\usepackage{longtable}
\usepackage{wrapfig}
\usepackage{rotating}
\usepackage[normalem]{ulem}
\usepackage{amsmath}
\usepackage{textcomp}
\usepackage{amssymb}
\usepackage{capt-of}
\usepackage{hyperref}
\usepackage{bm}
\author{Lukas Krenz}
\date{\today}
\title{Exposé: \\ Integration of Prior Knowledge for Regression and
  Classification with Sparse Grids }

\begin{document}
\maketitle

\section{Sparse Grids for Regression}
Sparse grids is a discretization technique that originates from numerical partial differential equations.
They have many applications for various problems. We concentrate here on regression.

Let \( \varphi_i(\bm{x})\) be a family of basis functions.

We now define
\begin{equation}
\boldsymbol{\phi}(\boldsymbol{x}) = \begin{pmatrix}
  \varphi_1(\bm{x}) \\
  \vdots \\
  \varphi_m(\bm{x})
\end{pmatrix}
, \quad
\boldsymbol{\Phi}(\boldsymbol{x}) = \begin{bmatrix}
  \boldsymbol{\phi}(x_1)^\intercal\\
  \vdots \\
  \boldsymbol{\phi}(x_n)^\intercal
\end{bmatrix},
\end{equation}
which we use to express our regression formula
\begin{equation}
\hat{y} = \sum_{j = 1}^m \alpha_j \varphi_{j}(\bm{x}) = \boldsymbol{\alpha}^\intercal \bm{\phi} (\bm{x})
\end{equation}
as a weighted sum of the basis functions.

We split out d-dimensional basis functions as follows using a tensor product
over several one dimensional functions
\begin{equation}
\varphi_j (x_1, \ldots, x_n) = \prod_{k=1}^d \varphi_{jk} (x_k).
\end{equation}
We can now formulate our goal as an optimisation problem of the form
\begin{equation}
\min_{\bm{\alpha}} \left\Vert  \bm{\Phi} \bm{\alpha} - \bm{y}   \right\Vert_2^2  + \lambda \mathcal{S}(\bm{\alpha}), 
\end{equation}
with \(\mathcal{S}(\bm{\alpha})\) as a regularisation operator and \(\lambda\) as a constant.
The standard regularisation method is the L2 regularisation 
\begin{equation}
\mathcal{S}(\bm{\alpha}) = \left\Vert \bm{\alpha} \right\Vert_2^2.
\end{equation}
From a Bayesian perspective it resembles a Gaussian prior on the weights with mean 0 and
a constant variance  and is therefore distributed as follows
\begin{equation}
\alpha \sim \mathcal{N} (0, \bm{I} \sigma^2 ).
\end{equation}

\section{Goals}
The goal of the proposed bachelor thesis is to extend the SG++ toolbox for sparse grids with more ways to include prior information
in the learning algorithm and to test different regularisation methods.

This main objective can be dissected into four smaller subgoals all of which will be implemented and tested for validity.

\subsection{Generalised weight prior}
The standard L2 regularisation assumes that all weights are distributed with the same variance. 

The goal is to enhance this in a way that enables us to assume a variance for
each individual weight and to use different and more general regularisation functions.

This means using a prior on the weights that's normally distributed with zero mean and covariance matrix \(S_0\).
To archive this, we use 
\begin{equation}
\mathcal{S} = \Vert L\bm{\alpha} \Vert_2^2
\end{equation}
for the regression term. In this equation \( L \) is a linear operator.

\subsection{Different regularisation methods: Lasso and Elastic net}
We can use lasso regression to replace our regularisation operator with 
\begin{equation}
\mathcal{S} = \Vert \bm{\alpha} \Vert_1,
\end{equation}
which uses the Manhattan norm.

Lasso regression can reduce some weights to zero. 
This results in an implicit feature selection and thus leads to sparsity.

Because the absolute value function is not differentiable at \(0\), we cannot use
the standard gradient descent optimisation method.
We must use a proximal gradient method, for example FISTA.

It is also possible to use both Lasso and L2-regression at the same time, this is called elastic net regression.

The goal is to implement lasso and elastic net regression and to test the quality of the results.
\subsection{Fixed-interaction-terms}
When using multiple regression variables, we have to choose our desired grid
resolution for each variable interaction.
For many problem categories we can use domain-specific knowledge. 

In image classification for example, we can treat each pixel as a feature.
The interaction of pixels that are directly next to each other are more
interesting to model with a high resolution than pixels with a higher distance.

The same strategy of modeling local interaction terms with a more finely
grained grid can be used for other problems, such as time series, as well.

The goal is to implement this method and to test,
whether we are still able to get solid results even when using a lower resolution for
some feature interactions.
\subsection{Subspace selection}
In addition to the interaction term selection we can also limit the number of
subspaces we use for the sparse grid. We can describe the sparse grid space
\(G^T_l\) of level \(l\) with
\begin{equation}
G^T_l = \bigcup_i G_i \text{ such that } \left\vert i \right\vert_1 - T \left\vert i \right\vert_\infty \leq (l + d - 1) - T l.
\end{equation}
where \(d\) represents the number of dimensions.
 
The constant \(T\) can be chosen in the interval \([0, \infty\)) and governs our
choice of subgrids. If we set \(T\) equal to \(0\) we recover the standard
sparse grid.
A higher value of \(T\) reduces the complexity of learning while reducing the
quality of our results.

The goal is to implement this subspace selection method and to test and
validate results obtained with different grid selections.
\end{document}