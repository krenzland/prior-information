\chapter{\abstractname}
This thesis discusses different ways of imposing prior knowledge about datasets on the sparse grid model for supervised learning.
We introduce a Tikhonov regularization method that uses information about the smoothness of the function we want to approximate.
We also present the sparsity-inducing penalties lasso, elastic net, and group lasso.
The different regularization approaches are compared with the standard ridge regularization.
Because some regularization penalties are not differentiable, we discuss the fast iterative shrinkage-thresholding algorithm and show how it can be used in conjunction with our added regularization methods.
Furthermore, we modify the grid generation procedure.
The first discussed method is generalized sparse grids, which allows us to control the granularity of the grid.
The second method is interaction-term aware sparse grids, which are used to construct smaller and more efficient grids for image recognition problems.
All methods were implemented with the \emph{SG++} library and showed promising results for both artificial and real-world datasets.

\section*{Zusammenfassung}
Diese Bachelorarbeit diskutiert verschiedene Wege, vorheriges Wissen über
Datensätze auf das Lernen mit dünnen Gittern zu übertragen.
Wir führen die Tikhonov Regularisierung ein, die Informationen über die
Glattheit der Funktion, die wir approximieren wollen, benutzt.
Wir präsentieren außerdem die Regularisierungsmethoden Lasso, Elastic Net und
Group Lasso, die alle zu dünnbesetzten Lösungsvektoren führen.
Die verschiedenen Methoden werden mit der klassischen Ridge-Regularisierung
verglichen.
Weil einige Funktionale nicht differenzierbar sind, diskutieren wir den ``Fast
Iterative Shrinkage Thresholding'' Algorithmus, und zeigen, wie er mit den neu
hinzugefügten Regularisierungsmethoden benutzt werden kann.
Desweiteren modifizieren wir den Gittergenerierungsalgorithmus.
Die erste Methode heißt verallgemeinerte dünne Gitter und erlaubt uns, die
Granularität der Gitter zu verändern.
Die zweite diskutierte Methode sind Gitter, die Wissen über Interaktionsterme
benutzen, was nützlich für bildverstehende Verfahren ist.
Alle Methoden wurden mit Hilfe der \emph{SG++} Bibliothek implementiert und zeigten
vielversprechende Ergebnisse für künstliche und echte Datensätze.
%%% Local Variables:
%%% mode: latex
%%% TeX-master: "../main"
%%% End:
