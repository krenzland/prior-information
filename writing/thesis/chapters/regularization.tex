\chapter{Regularization Methods\label{cha:regularization}}
During the discussion of \cref{eq:optGoal} we neglected the regularization
operator \(\mathcal{S} \from \mathbb{R}^d \to \mathbb{R}\).
In this chapter we focus on this operator exclusively.
We start with a short introduction to statistical regularization theory and then compare two groups of regularization methods:
Tikhonov regularization and sparsity-inducing penalties.

We will use the \(l_p\)-norms for vectors
\begin{align*}
  \Vert \bm{\alpha} \Vert_p & = \left (\sum_{a \in \bm{\alpha}}  \vert a \vert^p \right)^{\frac{1}{p}},\\
  \lim_{p \to \infty} \Vert \bm{\alpha} \Vert_p & = \max_{a \in \bm{\alpha}} a
\end{align*}
in this chapter.

\section{Regularization Theory}
Regularization helps us to train models that not only fit the data but also generalize well.
\sidetitle{Bias-Variance trade-off}
To understand how regularization works, it is helpful to decompose our error functional.
Our model assumes that there is a relation between the predictors \(\bm{x}\) and the target variable \(y\) that can be expressed as
\begin{equation*}
  y = f(\bm{\alpha}) + \varepsilon,\, \varepsilon \sim \mathcal{N}(0, \sigma^2),
\end{equation*}
where \(f(\bm{\alpha})\) is the function we want to approximate and \(\varepsilon\) is normally distributed noise.
Consider the expected prediction error at a point \(\bm{x}\)~\cite{esl}:
\begin{align}
  \label{eq:bias-variance}
  \text{Pred.~error}\left( \bm{x} \right) = \Exp \left[ \left(\bm{y} - \hat{f}(\bm{x}) \right)^2 \right] &=
              \Bias\left[\hat{f}(\bm{x}) \right]^2 + \Var\left[\hat{f}(\bm{x})\right] + \sigma^2 \intertext{with}
  \Bias \left[ \hat{f} (\bm{x}) \right] &= \Exp \left[\hat{f}(\bm{x}) - f(\bm{x})\right], \nonumber \\
  \Var \left[ \hat{f} (\bm{x}) \right] &= \Exp \left[ \hat{f}(\bm{x}) - \Exp \left[ \hat{f}(\bm{x}) \right]^2 \right] \nonumber,
\end{align}
where \(\hat{f}(\bm{x})\) denotes our approximation of the real function \(f(\bm{x})\).
We call \cref{eq:bias-variance} the bias-variance trade-off.
This equation splits the error into three different parts:
\begin{description}
\item[Bias] is the error caused by assumptions the model makes,
\item[Variance] is the fluctuation of the model around the mean,
\item[Irreducible Error (\(\sigma\))] is the error caused by noise that is inherent to the relation between the predictors and the target variable.
\end{description}

\sidetitle{Occam's razor}
Following the principle of Occam's razor---parsimonious models are better---all regularization methods penalize complexity.
Using regularization leads to smaller and simpler models, which increases the bias.
Of course, increasing this part of the error term makes no sense, if we would not get a payoff.
Regularization decreases the variance, thus making our model more robust.
In this chapter we consider regularization methods that have both a scaling
parameter, that controls the strength of our simplicity assumptions, and (sometimes) parameters that we use to modify the kind of our assumptions.
We call the parameter that controls the regularization strength \(\lambda\).

\sidetitle{Finding \(\lambda\)}
The choice of this parameter is important:
If we increase \(\lambda\), we exchange more bias for variance.
This is why \cref{eq:bias-variance} implies a trade-off---we cannot have the cake and eat it too!
We usually find the ideal \(\lambda\) by a more-or-less intelligent trial and error process.
To do this, we train a predictor for each \(\lambda\) we want to consider on a
subset of our data (called the training set), and test its performance using
cross-validation (\textsc{cv}). 

\sidetitle{Prior Information}
Another way to reason about regularization is as a method to encode our
assumptions directly into the training process.
Many regularization methods---and all mentioned in this chapter---can be
viewed from a Bayesian perspective, which gives an intuitive explanation of the
effect of our assumptions.

The choice of the regularization functional is crucial.
We do not know which one performs best without training a model.
Sometimes we can encode knowledge about the dataset structure but this is often
quite difficult.
Each method encodes different assumptions but all have in common that they
decrease the model complexity.

\section{Tikhonov Regularization}\label{cha:tikh}
Tikhonov regularization is one of the most commonly used regularization methods for ill-posed problems.
It is also widely used to regularize regression, which leads to solutions with a larger bias but with a smaller variance.
We will show the general form of this penalty first and then adapt the penalty to sparse grid learning.

\subsection{Theory}
We can use Tikhonov regularization by setting the regularization penalty
\(\mathcal{S}\) in \cref{eq:optGoal} to
\begin{equation}\label{eq:tik-langrangian}
\mathcal{S} = \Vert \bm{\Gamma} \bm{\alpha} \Vert_2^2,
\end{equation}
where \(\bm{\Gamma}\) is a linear operator.
The overall optimization goal is then given by
\begin{equation}\label{eq:tik-optGoal}
 \min_{\bm{\alpha}} \Vert \bm{\Phi} \bm{\alpha} - \bm{y} \Vert_2^2 + n \lambda \Vert \bm{\Gamma} \bm{\alpha} \Vert_2^2.
\end{equation}
We can also view this problem in the constraint minimization form
\begin{align}\label{eq:tik-constrained}
 \text{minimize} \quad &
 \left\Vert  \bm{\Phi} \bm{\alpha} - \bm{y}  \right\Vert_2^2 \nonumber \\
\text{subject to} \quad &  \Vert \bm{\Gamma} \bm{\alpha}  \Vert_2^2 \leq l,
\end{align}
for a certain constant \(l\). 
We can see from this formulation that Tikhonov regularization forces our scaled weights \(\bm{\alpha}\) to lie inside a \(d\)-dimensional sphere with a diameter of length \(l\).
There is an one-to-one correspondence between \(l\) in~\cref{eq:tik-constrained} and \(\lambda\) in~\cref{eq:tik-optGoal}.
Both variables determine the constraints.
Tikhonov regularization fits into a Bayesian framework.
We can interpret it as a multivariate-Gaussian prior on the weights with zero mean and covariance matrix \(\bm{\Gamma}^{-1}\)~\cite{stat-inverse}.
The prior is thus distributed as
\begin{equation}\label{eq:diagonal-prior}
\bm{\alpha} \sim \mathcal{N} (0, \bm{\Gamma}^{-1}).
\end{equation}

A common choice for \(\bm{\Gamma}\) is the identity matrix as proposed in~\cite{spatAdaptGrid}.
This corresponds to a penalty on the summed squared values of the weights.
It is a Gaussian prior with the identity matrix as its covariance matrix.
In statistics, this method is called ridge regression~\cite{esl}.

\sidetitle{Diagonal Matrix}
The identity Tikhonov matrix assumes that all weights are distributed with the same variance.
Luckily, we can do better for the sparse grid method.
Following the assumption of \cref{eq:coefficients-h2mix}, we can express every
function \(f \in H_2^{\text{mix}}\) as a weighted sum of basis functions.
For these functions the following upper bound on the hierarchical coefficients \(\alpha_{\bm{l, i}}\) holds:
\begin{equation}\label{eq:weights-bound}
  \vert \alpha_{\bm{l, i}} \vert \leq 2^{-d -2 \vert \bm{l} \vert_1} \cdot \Vert D^{\bm{2}} f \Vert_{\infty} 
                              \in \BigO \left( 2^{-2 \vert \bm{l} \vert_1} \right),
\end{equation}
where the differential operator norm only depends on the function \(f\) and neither on the dimension nor the level of the basis functions~\cite{bungartzSparse}.
Because the dimension is constant for a given grid we can safely exclude it.

This fact can be used to implement a regularization method that is better suited for functions in \(H^2_\text{mix}\).
We impose an improved prior on the weights using Tikhonov regularization
with the matrix
\begin{equation}\label{eq:diagonal-matrix}
\bm{\Gamma}_{i,i} = c^{\vert \bm{l} \vert_1 - d},
\end{equation}
for a constant \(c\)~\cite{sparse-parsimony}.
For \(c = 4\) this corresponds to a prior on the variance of the weights that is identical to the upper bound given by \cref{eq:weights-bound} up to a multiplicative constant.
A different \(c\) can be used as well, we can either treat \(c\) as an inherent property of the method or as an additional hyper-parameter.
We use the dimension \(d\) as a normalizing factor, this way the prior
corresponds to the series \((1, \nicefrac{1}{4}, \nicefrac{1}{16}, \ldots)\).
The resulting prior is depicted by \cref{fig:diagonal} for a two dimensional grid.

\begin{figure}[htb]
  \centering
  \includegraphics[width=0.7\textwidth]{diagonal}
\caption[Diagonal Regularization]{
Prior generated by \cref{eq:diagonal-matrix} for a two dimensional grid with level four.
Each number is centered on a grid point and
corresponds to the prior for that particular weight.
}\label{fig:diagonal}
\end{figure}

\FloatBarrier{}

\subsection{Implementation}
Each regularization method that can be used with the conjugated gradient solver
is implemented in the \emph{SG++} library by specializing the base class
\emph{OperationMatrix} that offers a method called \textsc{mult}, which accepts a weight
vector and returns a scaled version of the weights.
The \emph{OperationMatrix} for the standard ridge regularization returns its input
weights unchanged.

For our implementation we created a class \emph{OperationDiagonal} that inherits
from \emph{OperationMatrix}.
The constructor accepts an argument that allows the specification of the
exponent base \(c\).
Its implementation of \textsc{mult} multiplies each input weight with the corresponding inverse prior.
We calculate the multiplier for each weight during the first call of the multiplication method and cache it until the grid size changes.
This means that we only need to perform this calculation once per refinement step.
We determine the multiplier for each grid point by simple iterating over all existing grid points and save the result of \cref{eq:diagonal-matrix}.
The cost of this operation is in \(\BigO(n)\) and is thus negligible.

The class \emph{OperationDiagonal} can then be used in the same way as the
implementation of the ridge regularization.

\subsection{Results \textit{\&} Discussion}\label{sec:tikh-discussion}
To prove the effectiveness of the diagonal method, we first show empirically that \cref{eq:weights-bound} holds for a simple function in \(H^2_{\text{mix}}\).
We then present results that indicate that our proposed regularization functional shows improved results for an artificial dataset, for which the upper bound for the surpluses holds by construction.
Finally we show benchmark results for two real-world datasets, comparing the identity matrix with the diagonal operator.

\begin{figure}[ht]
  \centering
    \includegraphics[width=1\textwidth]{parabola_contour}
  \caption{Surface and contour plot of the inverse parabola function~\ref{eq:inverse-parabola}}
\end{figure}

\sidetitle{Inverse Parabola}
We first consider the inverse parabola
\begin{equation}\label{eq:inverse-parabola}
  f(x_1, x_2) = 16x_1(1-x_1)x_2(1-x_2).
\end{equation}
We created a two dimensional grid with standard linear basis functions and level
three that had 17 grid points.
The construction also works for a different choice of level.
The dataset was then created by setting the features \(x_1, x_2\) equal to the coordinates of one grid point each, the target \(\bm{y}\) was then calculated using \cref{eq:inverse-parabola}.
The sparse grid regression model trained on this dataset and the aforementioned
grid was able to recover the target perfectly, with a mean squared error smaller
than the machine epsilon.
More interestingly, the calculated weights were identical to our prior.
Note that we did not perform regression but rather interpolation.
While this does not prove that this prior holds, it is a simple example for this construction.

\sidetitle{Recovering Weights}
But does our prior improve the results for a function, when we include our prior
knowledge about the weights bound?
To test this hypothesis, we constructed another artificial dataset.
We first created a two dimensional sparse grid learner with level 3 and sampled
its weights from the normal distribution \(\bm{\alpha} \sim \mathcal{N}(0,
\bm{\Gamma}^{-1})\), which corresponds to the prior~\ref{eq:diagonal-prior}.
The operator \(\bm{\Gamma}\) is our diagonal matrix from \cref{eq:diagonal-matrix}.
Let \(\bm{X} \in \mathbb{R}^{n \times 2}\) be our design matrix, where each row is drawn from a two-dimensional uniform distribution.
We then created our target vector \(\bm{y}\) by predicting the result of
\(\bm{X}\) using the constructed model.
Right now, this gives a trivial regression problem and to show that the diagonal
matrix yields better results, we need to add some noise to \(\bm{y}\).
Let \(\sigma\) denote the standard deviation of \(\bm{y}\).
We then crafted different variants of our dataset by adding normal noise to the
target variable with mean zero and standard deviation \(s \sigma\), for some
values of \(s\).
The signal-to-noise-ratio (\textsc{snr}) of the modified target is then given by \(1/s\).

\begin{table}[htb]
\centering
\begin{tabular}[c]{S[table-format=1.1]
  S[table-format=1.1]
  S[table-format=1.4, table-figures-exponent=2, table-sign-mantissa, table-sign-exponent]
  S[table-format=2.6]
  S[table-format=2.6]
  S[table-format=2.6]}
  \toprule \multicolumn{1}{r}{\textsc{snr}}
& \multicolumn{1}{c}{Exponent Base}
& \multicolumn{1}{c}{\(\lambda\)}
& \multicolumn{1}{c}{Weights-\textsc{rmse}}
& \multicolumn{1}{c}{\textsc{Cv}-\textsc{rmse}}
& \multicolumn{1}{c}{\(\sigma_\text{noise}\)}
\\\midrule
4.0 & 4.0 & 4.7149e-05 & 0.062738 & 0.274558 & 0.262597\\
2.0 & 4.5 & 1.4563e-04 & 0.080263 & 0.545097 & 0.525194\\
1.0 & 6.5 & 2.5595e-04 & 0.095696 & 1.083762 & 1.050387\\
0.5 & 4.5 & 2.4421e-03 & 0.109284 & 2.159250 & 2.100775\\
\bottomrule
\end{tabular}
\caption[Best exponent bases for diagonal test dataset.]{
Combinations of \(\lambda\) and exponent base for different \textsc{snr}s that
achieved the best root-mean-square-error (\textsc{rmse)}.
}\label{fig:diagonal-testdata}
\end{table}

We performed a grid search\footnote{
We used a Monte-Carlo cross-validation method with ten iterations and
a 9:1 train-validation split for this experiment, in contrast to the usual
10-Fold method, to compare the different estimators.
} for sparse grid estimators, testing on a grid
consisting of different choices for \(\lambda\) and the exponent bases in the
interval \([2, 10]\) with stepsize 0.5.
The results for the best parameters can be seen in \cref{fig:diagonal-testdata}.
Note that we achieved the best cross validations errors with exponent bases
that are close to four.
There is one outlier, the \(\textsc{snr} = 1\) case, which can be explained
by the very noisy data.
The regularization parameter \(\lambda\) and the \textsc{rmse} for the weights
decrease for a higher \textsc{snr}, as expected.
All results are close to the theoretical optimal error, which is equivalent to the variance of
the noise.
Those results indicate that our proposed regularization method improves performance for a dataset, which can be approximated by a model with surpluses that follow the upper bound.

\sidetitle{Concrete Dataset}
So far we have only seen examples for artificial datasets, which adhered to our assumptions by construction.
We now introduce our first real-world dataset, the concrete dataset.
For this dataset our goal is to predict the compressive strength of concrete,
using the recipe of its mixture and its age in days.
The recipe consists of seven quantitative predictors, all given in the unit
kilogram per thousand liters.
Concrete consists of the ingredients cement, blast furnace slag, fly ash,
water, superplasticizer, coarse aggregate, and fine aggregate.
The dataset was donated to the \textsc{uci} machine learning
repository~\cite{datasets-uci} by \citeauthor{datasets-concrete} and was first
published in their paper~\cite{datasets-concrete}.
It consists in 1030 instances altogether.
We split the data and used 80\% for training and the other 20\% solely as
testing data.

We performed a grid search using a learner with level four for the diagonal
regularization with fixed exponent bases \(c = 4\), and \(c = 1\), which
correspond to the diagonal and the identity matrices respectively.
Each learner performed five adaptivity steps, each refining three grid points.
The performance of the estimators was estimated using a standard 10-fold cross validation procedure.

\begin{figure}[htb]
  \centering
  \includegraphics[width=\textwidth]{tikhonov_concrete_l4}
  \caption{Results for the concrete dataset obtained with estimators with level
    four for two different Tikhonov matrices.}
  \label{fig:tikhonov-concrete-l4}
\end{figure}

The results are shown in \cref{fig:tikhonov-concrete-l4}.
We can see that our method resulted in better results than the identity
regularization, although the difference between the two methods was small.

\sidetitle{Power Plant Dataset}
We also tested the performance for the power plant dataset, for which a visualization is given by \vref{fig:tsne-power-plant}.
The target variable of this dataset is the hourly energy output for a combined
cycle power plant.
To predict this target, we use the temperature, the ambient pressure, the
relative humidity, and the exhaust vacuum as predictors.
This dataset appeared first in~\cite{datasets-powerplant} and was donated to the
\textsc{uci} machine learning repository~\cite{datasets-uci} by \citeauthor{datasets-powerplant}.
It consists in 9568 instances, which were split into a training and testing
dataset at a ratio of 8:2.

We then performed a grid search over an grid of lambdas in the interval
\([10^{-10}, 10^{-1}]\) for learners with level 5, again using ten-fold cross validation.
The grids were refined five times, refining three points for each adaptivity step.
The results can be seen in \cref{fig:tikhonov-power-plant-l5}.
Note that this figure only shows the results in a small interval, all values of
\(\lambda\) that were larger than the values shown resulted in far larger
errors.
Again, we can see that the diagonal regularization improved the \textsc{rmse} by
small margin of roughly 0.03.
Note that this improvement is larger than the overall-effect of the identity
regularization, which implies that, in contrast to the ridge regularization, the
diagonal matrix achieved significant improvements over the non-regularized regression.

\begin{figure}[htb]
  \centering
  \includegraphics[width=\textwidth]{tikhonov_power_plant_l5}
  \caption{Identity vs.~diagonal matrix, for the power plant dataset with
    learners of level five.}
  \label{fig:tikhonov-power-plant-l5}
\end{figure}

We can conclude from these results that the diagonal regularization method is
able to result in better outcomes if the datasets adhere to the assumptions of
the method.
The tests on real-world datasets showed that our proposed method is a solid alternative
to the standard ridge regularization penalty, increasing the performance by a
small margin for a negligible additional performance cost.
\begin{figure}[htb]
  \centering
  \includegraphics[width=\textwidth]{tsne_power_plant}
  \caption{
2-dimensional visualization of the power plant dataset. This visualization was generated using the t-Distributed Stochastic Neighbor Embedding (\textsc{t-sne}) algorithm~\cite{tsne}. Data points close to each other in the
  original, multi-dimensional dataset are also close in this two-dimensional
  representation. The target values are not considered for this calculation, they
are only used to determine the color of the points. The axes contain no useful
information and are therefore omitted.
}
  \label{fig:tsne-power-plant}
\end{figure}
\FloatBarrier{}
\section{Sparsity-Inducing Penalties}\label{sec:sparse-penalties}
We have seen different variations of Tikhonov regularization.
All regularization methods so far have one thing in common: they used the squared Euclidean-norm.
In this section we will look at three different methods, which all use different norms that induce sparsity.
Sparsity in this context means that some entries of the weight vector \(\bm{\alpha}\) are exactly zero.

\begin{figure}[tb]
    \centering
    \includegraphics[width=1\textwidth]{constraints}
    \caption{Constraint regions for lasso and ridge regularization respectively. The red contours correspond
      to a \(l_2^2\) loss function with optimal solution \(\bm{\alpha}^*\). The
      figure is inspired by figure 3.11 of~\cite{esl}.}\label{fig:reg-constraints}
\end{figure}

\sidetitle{Lasso}
A first simple method is the so called lasso\footnote{In the original
  paper~\cite{lasso} the name lasso  was introduced as an
  acronym for ``least absolute shrinkage and selection operator''.
We use the term in a more metaphorical manner, where the lasso stands for an
actual rope used to catch cattle, or in our case predictors. See also~\cite{sparse-learning}.},
first published by Tibshirani in~\cite{lasso}.
We can represent this procedure in a form similar to \cref{eq:tik-constrained} for a constant \(l\):
\begin{align*}\label{eq:lasso-constrained}
\text{minimize} \quad &
 \left\Vert  \bm{\Phi} \bm{\alpha} - \bm{y}  \right\Vert_2^2 \\
\text{subject to} \quad & \Vert \bm{\alpha}  \Vert_1 \leq l,
\end{align*}
which we can also cast into the more convenient Lagrangian representation
\begin{equation*}
\mathcal{S} = \Vert \bm{\alpha} \Vert_1.
\end{equation*}
We can see from the constraint form that the lasso only accepts solutions that
are inside a \(d\)-dimensional hyper-cube centered on the origin.
\Cref{fig:reg-constraints} compares the constraint regions of the ridge and
lasso regularization.
Let \(\hat{\bm{\alpha}}\) denote the non-regularized least squares solution.
All values of \(l \geq \Vert \hat{\bm{\alpha}} \Vert_1\) shrink the predictors,
some weights can be exactly zero~\cite{lasso}.

\begin{figure}[tb]
    \centering
    \includegraphics[width=0.7\textwidth]{pdfs}
    \caption{Plot of priors for ridge, lasso and elastic net.}\label{fig:reg-pdfs}
\end{figure}

\sidetitle{Laplace Prior}
We can also view this smoothness function from a Bayesian perspective.
In this context, lasso regularization can be seen as a Laplace prior on the
weights with zero mean.
A visualization of the prior for ridge and lasso regularization can be seen in \cref{fig:reg-pdfs}.
We can see that the Laplace prior puts more of its weight at zero and on its
tails than the normal distribution, which implies that solutions are more likely
to be exactly zero or larger than for the ridge estimate~\cite{lasso}.

\sidetitle{Relation to Best-Subset}
The lasso is an instance of a large family of regularization functionals, where the penalty is realized as the \(l_p\) norm of the weights.
We define the ``\(l_o\)-norm'' \(\Vert \bm{\alpha} \Vert_0 = \vert \{a \in
\bm{\alpha} \, | \, a \neq 0\} \vert\) as the cardinality of the support of \(\bm{\alpha}\), i.e.~the number of entries of a vector that are not zero.
The name ``\(l_o\)-norm'' reflects that it is similar to an \(l_p\) norm but is
not a proper norm itself.
The \(l_o\)-penalty corresponds to the standard best-subset feature-selection method.
Because the \(l_p\) norms are only convex for \(p \geq 1\), the lasso can be interpreted as the best convex approximation of the best-subset method~\cite{sparse-learning,lasso}.

\sidetitle{Elastic Net}
After we convinced ourselves that the lasso indeed leads to sparse
solutions, we will discuss its weaknesses.
An important modification of the lasso is the elastic net penalty, first introduced in~\cite{elasticnet} by \citeauthor{elasticnet}.
Let \(n\) denote the number of data points and \(p\) the number of grid points. The lasso does not show good results in the following situations~\cite{elasticnet}:
\begin{itemize}
\item In the (\(n < p\))-case, the lasso selects at most \(n\) predictors.
\item When the predictors are highly correlated, the lasso selects one of the
  predictors at random or shows otherwise unstable behaviour.
  If two variables are identical, the lasso solution is not unique~\cite{sparse-learning}.
\item The Tikhonov regression shows better practical results than the lasso in the
  correlated case.
\end{itemize}
All those reasons indicate that the lasso is not a good choice for some problems.
The Tikhonov regression is no direct competitor, because it does not lead to
sparse solutions.

The elastic net solves those problems by adding some ridge regularization to the
lasso penalty.
It is a compromise between both methods and is given by
\begin{equation}
  \label{eq:elastic-net}
  \lambda \mathcal{S}(\bm{\alpha}) = \lambda \Vert \bm{x} \Vert_2 +  \gamma \Vert \bm{x} \Vert _1,
\end{equation}
where \(\lambda\) and \(\gamma\) are two independent parameters~\cite{elasticnet}.
Although this form is more convenient for some of the following calculations, it does
not yield a nice interpretation.
This is why we re-parametrize the previous equation as
\begin{equation*}
  \lambda \mathcal{S}(\bm{\alpha}) = \lambda_1 \left( \left(1 - \lambda_2 \right) \Vert \bm{x} \Vert_2  + \lambda_2 \Vert \bm{x} \Vert _1\right),
\end{equation*}
where \(\lambda_1\) determines the overall regularization effect and \(\lambda_2\) controls the relative influence of the lasso term.
From this equation we can recover both the ridge and the lasso regularization, by setting \(\lambda_2\) to \(0\) or \(1\) respectively.
The advantage of the method is then obvious.
While the lasso part of the penalty enforces sparsity, the ridge part shrinks
correlated variables together, thus stabilizing the feature selection~\cite{elasticnet}.
A visualization of the corresponding prior, a mixture of an Gaussian and a Laplace prior, can be seen in~\cref{fig:reg-pdfs}.

\sidetitle{Group Lasso}
Another useful generalization of the lasso is the group lasso that shrinks groups of weights at the same time.
Either all members of a group are selected, or none at all.
It was first developed by \citeauthor{grouplasso} in~\cite{grouplasso}, then
extended to the logistic regression method in~\cite{grouplasso-logistic}.
\citeauthor{grouplasso-generalizations} discussed some generalizations of the
method in~\cite{grouplasso-generalizations} and a discussion of the statistical
properties is offered by~\cite{grouplasso-benefit}.
The discussion here follows~\cite{sparse-learning}, unless otherwise noted.

Let \(\mathcal{P}\) denote a partition of \(\bm{\alpha}\), i.e.~a set of disjoint subsets whose union is \(\bm{\alpha}\).
We can then define the group lasso as
\begin{equation*}
  \mathcal{S}({\bm{\alpha}}) = \sum_{\bm{p} \in \mathcal{P}} \left(\sqrt{\vert \bm{p} \vert}\right) \Vert  \bm{p} \Vert_2,
\end{equation*}
where \(\vert p \vert\) denotes the size of a group \(p\) and is used as a weighting factor.
We can choose to weight the groups differently but the square root of the cardinality is a useful factor that is simple to calculate and works well in practice.
If we would not include this factor larger groups would be more likely to be included in the final model~\cite{sparse-learning}.

\sidetitle{Order of grid points}
Let
\begin{equation*}
  \operatorname{order}(\bm{p}) = \vert \{ i\, \mid p_i \neq 0.5 \} \vert
\end{equation*}
denote the cardinality of the support for a grid point with coordinates \(\bm{p}\).
A modified linear basis function with coordinate 0.5 for a dimension is constant with respect to
this dimension.
For example, the bias term is constant for all dimensions and has therefore
order zero.
The points of order one correspond to all basis functions that are constant for
all but one dimension, and so on.
We call all grid points with order larger than one interaction terms, because
they model the interaction between different dimensions.

\sidetitle{Grouping Grid Points}
We then partition our weights into groups consisting of all terms of the same order.
This grouping corresponds to the original predictors, including the interactions
between them.
\Cref{alg:group} shows a possible algorithm which results in our chosen
partition.
We use the fact that all grid points have a unique sequence number, which we can use to refer to a specific basis function.
This way of grouping variables and its usefulness for the group lasso is also
discussed in~\cite{sparse-parsimony}.
Note that we can recover the original lasso penalty by choosing partitions of size one.

\begin{algorithm}[h]
\caption{Group Lasso: Group}\label{alg:group} 
 \begin{algorithmic}[1]
   \Require{gridStorage that contains all grid points, weight vector
     \(\bm{\alpha}\).}
   \Statex
   \Function{group}{gridStorage, $\bm{\alpha}$} 
    \Let{curGroup}{0}
    \Let{groups}{HashMap<vector<bool>, int>()}
    \Let{groupVec}{vector<bool>()}
    \For{point \(\in\) grid}
      \Let{usedDims}{vector<bool>(false,\ldots, false)}
      \For{\(\text{curDim} \in \{ 0, 1, \ldots, d\}\)}
        \Let{coordinate}{\Call{getCoordinate}{point, curDim}}
        \Let{usedDims[curDim]}{coordinate \(\neq\) 0.5}
      \EndFor
      \If{usedDims \(\in\) groups}
        \Let{groupVec[\Call{getSeqNumber}{point}]}{groups[usedDims]} 
        \Else
          \Let{groupVec[\Call{getSeqNumber}{point}]}{curGroup}
          \Let{curGroup}{curGroup + 1}
      \EndIf
    \EndFor
   \State \Return{groups, groupVec}
   \EndFunction
 \end{algorithmic}
 \end{algorithm}

It can be shown that the group lasso penalty performs better than the lasso
regularization if the group structure is evident in the data.
If the structure is not contained in the data, the lasso shows stronger
theoretical results.
For a more elaborate discussion of these and more results of the group lasso, we refer to~\cite{grouplasso-benefit}.

\sidetitle{Non-differentiable Penalties}
All discussed sparsity-inducing penalties are not differentiable at zero.
We were able to solve the Tikhonov regularization method using a standard
conjugated gradient scheme.
This is not possible for the methods presented in this section, because we cannot rely on gradient information any more.
This is why we need to solve these problems using a gradient-free optimization procedure.
Because we can still profit from the structure of our problem, we do not have to use a black-box-optimization algorithm.
We present a solver for least-squares problems with sparse regularization in the following section.

\subsection{Proximal Methods}
There are many solvers for the lasso and related methods.
In this section we present the Fast Iterative Shrinkage Tresholding Algorithm (\fista), first introduced in~\cite{fista}.
This solver offers a good compromise between flexibility and performance and we
can use it to solve all presented sparse regularization methods.
The discussion of \fista\ follows~\cite{fista}, the description of
the proximal operators follows~\cite{proxsurvey}.
\fista\ is able solve problems of the form
\begin{equation}\label{eq:composite-goal}
\min_{\bm{\alpha}} F(\bm{\alpha}) = f(\bm{\alpha}) + g(\bm{\alpha}),
\end{equation}
i.e.~minimizing functions that can be expressed as a sum of a convex, smooth function \(f(\bm{\alpha})\) and another convex, possibly non-smooth function \(g(\bm{\alpha})\).
\sidetitle{Lipschitz constant}
It is required that \(f(\bm{\alpha})\) has a Lipschitz-continuous gradient. 
This means that the following condition holds for all possible vectors \(\bm{x}\) and \(\bm{y}\) for some positive constant \(L\):
\begin{equation}   \label{eq:lipschitz}
 \Vert \bm{\nabla} f(\bm{x}) - \bm{\nabla} f(\bm{y}) \Vert \leq L \Vert \bm{x} - \bm{y} \Vert.
\end{equation}
We call the smallest possible value of \(L\) the Lipschitz constant.

For our goal given by \cref{eq:composite-goal} we set \(f(\bm{\alpha})\) to
\begin{equation*}
 f(\bm{\alpha}) = \frac{1}{2} \left\Vert  \bm{\Phi} \bm{\alpha} - \bm{y}   \right\Vert_2^2,
\end{equation*}
such that the gradient of \(f\) is then given by
\begin{equation*}
  \bm{\nabla} f(\bm{\alpha}) = \bm{\Phi}^\intercal \left(\bm{\Phi} \bm{\alpha} - \bm{y} \right).
\end{equation*}
The Lipschitz constant for the gradient of \(f\) is
\begin{equation}
  \label{eq:lipf}
  L_{\bm{\nabla} f} = {\left(\sigma_{\max} \left(\bm{\Phi}\right)\right)}^2,
\end{equation}
where \(\sigma_{\max} \) corresponds to the maximum singular value~\autocite{fista}.
Additionally we set \(g\) to
\begin{equation*}
  g(\bm{\alpha}, \lambda) = \mathcal{S} \left( \bm{\alpha}, \lambda \right).
\end{equation*}

\sidetitle{Moreau envelope}
We now define the Moreau envelope of a function \(g (\bm{\alpha})\), which is given for any \(\lambda \in (0, +\infty)\) by
\begin{equation}
  \label{eq:moreau-reg}
  M_{g}(\bm{\alpha}, \lambda) = \inf_{\bm{x}} \left\{  g(x) + (1/(2\lambda)) \Vert x - \bm{\alpha} \Vert_2^2 \right\}.
\end{equation}
It is a regularized, smooth version of our function \(g(\bm{\alpha})\) that has the
same minimum as \(g(\bm{\alpha})\).
That means, that every point that minimizes \(M_{g}(\bm{\alpha}, \lambda)\) also
minimizes our original function \(g (\bm{\alpha})\)~\cite{proxsurvey}.
Finally, we are able to define the proximal operator that returns the infimum
point of \cref{eq:moreau-reg} by
\begin{equation}
  \label{eq:proximal}
  \prox[\(g\)]{\bm{\alpha}}{\lambda}= \argmin_{\bm{x}} \left\{ g (\bm{x}) + (1/ (2 \lambda)) \Vert \bm{x} - \bm{\alpha} \Vert_2^2 \right\}.
\end{equation}
\sidetitle{Proximal Operator}
The proximal operator can be viewed as a gradient step with stepsize \(\lambda\) on the Moreau envelope
\(M_g(\bm{\alpha}, \lambda)\)
\begin{equation*}
  \prox[\(g\)]{\bm{\alpha}}{\lambda} = \bm{\alpha} - \lambda \bm{\nabla} M_{g}(\bm{\alpha}, \lambda).
\end{equation*}
This identity follows by rewriting \cref{eq:moreau-reg} in terms of the proximal operator and calculating the gradient~\cite{proxsurvey}.
We can use \cref{eq:proximal} to minimize \(M_g(\bm{\alpha}, \lambda)\) and
thus also for optimizing \(g(\bm{\alpha})\).
In the most general case \cref{eq:proximal} would imply the need to solve a convex optimization problem.
Fortunately we can find closed form solutions for many functions.
Consider for example \(g(\bm{\alpha}) = 0\).
In this case the Moreau envelope and the proximal operator are trivial
\begin{align*}
 M_{0}(\bm{\alpha}, \lambda) &= \inf_{\bm{x}} \left\{ 0 + 1/(2 \lambda) \Vert \bm{\alpha} - \bm{x} \Vert_2^2 \right\}, \\
 \prox[\(0\)]{\bm{\alpha}}{\lambda} &= \bm{\alpha}.
\end{align*}
It is obvious that the minimum of \(M_g(\bm{\alpha}, \lambda)\) is equivalent to the minimum of \(g(\bm{\alpha})\), the proximal operator is corresponds to a gradient step on \(M_g\).

We are now going to develop a minimizer for our composite goal that resembles a majorization-minimization algorithm.
To do this, we first define an upper-bound of \(F(\bm{\alpha})\)~(\emph{majorizing}) that we are then going to minimize~(\emph{minimization})~\autocite{proxsurvey}.

\sidetitle{Upper Bound}
We first give a regularized linearization of \(f(\bm{\alpha})\) at an arbitrary,
but fixed point \(\bm{y}\) for an \(L > 0\):
\begin{equation}\label{eq:f-approx}
  \hat{f}_L(\bm{\alpha}, \bm{y}) = f(\bm{y}) + \left< \bm{\alpha} - \bm{y}, \bm{\nabla} f (\bm{y}) \right> +
  L/2 \Vert \bm{\alpha} - \bm{y} \Vert_2^2,
\end{equation}
where the angle brackets \( \left< \bm{x}, \bm{y} \right> = \bm{y}^\intercal \bm{x} \) represent the inner product.
The first two summation terms are given by the first order Taylor expansion of \(f(\bm{\alpha})\) at the point \(\bm{y}\), the last term can be interpreted as a trust-region or regularization, that punishes large deviations from~\(\bm{y}\)~\cite{proxsurvey}.
We then combine this linearization with our second function to archive an upper-bound of \(F(\bm{\alpha})\):
\begin{equation}\label{eq:goal-approx}
  Q_L(\bm{\alpha}, \bm{y}) = f(\bm{y}) + \left< \bm{\alpha} - \bm{y}, \bm{\nabla} f (\bm{y}) \right> +
  L/2 \Vert \bm{\alpha} - \bm{y} \Vert_2^2 +
  g(\bm{\alpha}).
\end{equation}
We can see from \cref{eq:lipschitz} that \(Q_L(\bm{\alpha}, \bm{y})\) is an
upper-bound of \(F(\bm{\alpha})\) iff.~L is equal to or greater than the Lipschitz
constant of \(\bm{\nabla} f(\bm{\alpha})\).

\sidetitle{Fixed-point minimizer}
The minimizer for this approximation is then given as the fixed-point equation
\begin{align}\label{eq:step}
  \pi_{g(\bm{\alpha})}(\bm{\alpha}^*, L) &=  \argmin_{\bm{x}} \left\{ Q_L(\bm{x}, \bm{\alpha}) \right\}\nonumber\\
       &= \prox[\(g\)]{\bm{\alpha^*} - L^{-1} \bm{\nabla} f (\bm{\alpha^*})}{L^{-1} } \nonumber\\
       &= \prox[\(g\)]{\bm{\alpha^*} - L^{-1} \bm{\Phi}^\intercal \left(\bm{\Phi} \bm{\alpha^*} - \bm{y} \right)}
         {L^{-1}},
\end{align}
where \(\bm{\alpha}^*\) denotes the optimal solution and \(L\) is the Lipschitz constant of \(\bm{\nabla} f\) given by \cref{eq:lipf}~\cite{fista}.
In this equation \(L\) is used to determine the optimal stepsize.
This minimizer is called proximal gradient algorithm (or proximal-splitting) in the literature, because we first perform a gradient step on \(f'_L(\bm{\alpha})\) given by~\ref{eq:f-approx} and then a proximal step on \(g(\bm{\alpha})\)~\cite{proxsurvey}.
Using \cref{eq:step} repeatedly on a point will result in the fixed-point, i.e.~the minimum of the upper bound, and thus also in the minimum of our original goal~\cite{proxsurvey}.

\begin{algorithm}
 \caption{Iterative Shrinkage Tresholding Algorithm (\ista)~\cite{fista}}\label{alg:ista} 
 \begin{algorithmic}[1]
   \Require{Lipschitz constant \(L\) of \(\bm{\nabla} f\), regularization parameter \(\lambda\)}
    \Statex
    \Function{Ista}{$L, \bm{\alpha}$} \Comment{\(\bm{\alpha}\) is an initial guess}
      \While{not converged}
        \Let{$\bm{\alpha}$}{\( \pi_{g(\bm{\alpha})} \left( \bm{\alpha}, L \right) \)}
      \EndWhile
     \State \Return{\(\bm{\alpha}\)}
    \EndFunction
\end{algorithmic}
\end{algorithm}

\Cref{eq:step} is all we need to define the simple iterative scheme called
Iterative Shrinkage Tresholding Algorithm~(\ista) as seen in \cref{alg:ista}.
Originally the name \ista\ was only used for solving the Lasso problem, but is now used for the more general algorithm as well. 
For our trivial function \(g(\bm{\alpha}) = 0\) this iterative scheme is identical to the standard gradient descent algorithm.

\sidetitle{Proximal Operators for Regularization}
So far we have only seen the proximal operator of a very simple function.
This is of course not satisfactory, the motivation for this chapter is solving the Lasso and the Elastic Net regularization problems.
Fortunately, closed form solutions for the other needed proximal operators exist as well:
\begin{align}
\label{eq:prox-lasso}
\text{for Lasso} \quad &&
    \left( \prox[\(\lambda \Vert \bm{\alpha} \Vert_1\)]{\bm{\alpha}}{t} \right)_i &= \left[\alpha_i - t \lambda \right]_+
    - \left[ -\alpha_i - t \lambda \right]_+, \\
\text{for Ridge} \quad &&
                          \left(  \prox[\( \lambda \Vert \bm{\alpha} \Vert_2^2\)]{\bm{\alpha}}{t} \right)_i &= (\alpha_i/(1 + 2t \lambda)), \nonumber\\
  \text{for Elastic Net} \quad && \prox[\( \lambda \Vert \bm{\alpha} \Vert_1 + \gamma \Vert \bm{\alpha} \Vert_2^2\)]{\bm{\alpha}}{t} &=
                                                                                                                                       \left( 1/(1 + 2 t \gamma) \right) \left(\prox[\(\lambda \Vert \bm{\alpha} \Vert_1\)]{\bm{\alpha}}{t}\right) \nonumber\\
  \text{for Group Lasso} \quad && {\left(\prox[\( \lambda \sum_{p \in \mathcal{P}} \sqrt{\vert p \vert} \Vert p \Vert_2 \)]{\bm{\alpha}}{t} \right)}_p &=
                                                                                     \left[ 1 - \left(\lambda t \sqrt{\vert p \vert}\right) \left( \Vert p \Vert_2 \right)^{-1} \right]_+ p \nonumber
\end{align}
where \( \left( x \right)_+ = \max(x, 0) \) denotes the positive part of \(x\)
and \(t\) is a stepsize.
The regularization parameters depend on the function \(g(\bm{\alpha})\).
We omit the derivations for the sake of brevity, the interested reader refers to the survey paper~\cite{proxsurvey}.
By setting the regularization parameter \(\lambda\) equal to zero, we again recover the gradient minimization method.

\sidetitle{Minimizing Lasso}
These proximal operators can be used to define a minimizer for a non-smooth
function \(g\).
For example, combining~\cref{eq:step} with the proximal operator for the
Lasso functional \(g(\bm{\alpha}) = \lambda \Vert \bm{\alpha} \Vert_1\) given by~\cref{eq:prox-lasso} results in
the minimizer
\begin{align*}
  \pi_{\lambda \Vert \bm{\alpha} \Vert_1}(\bm{\alpha^*} ,L)
  &=  \prox[\(\lambda \Vert \bm{\alpha} \Vert_1\)]{\bm{\alpha^*} - L^{-1} \bm{\nabla} f (\bm{\alpha^*})}{L^{-1}} \\
  &= {\left[ \left(\bm{\alpha^*} - L^{-1} \bm{\nabla} f (\bm{\alpha^*}) \right) - \lambda L^{-1} \right]}_+ -
    {\left[ -\left(\bm{\alpha^*} - L^{-1} \bm{\nabla} f (\bm{\alpha^*}) \right) - \lambda L^{-1} \right]}_+,
\end{align*}
again given as a fixed-point iteration.
In this equation the function \( \left[ x \right]_+ \) is applied element-wise on its input vector.
We can then use this minimizer with \cref{alg:ista} to compute a solution to
the Lasso problem.
The fixed-point equations for the Lasso and Elastic Net Regularization are
computed analogously.

\sidetitle{Nesterov's accelerated gradient descent}
\ista always converges to the global maximum, but only does so linearly~\cite{fista}.
To overcome this problem, Beck and Teboulle combined the \ista algorithm with
the accelerated gradient descent algorithm discovered by Nesterov. 
Nesterov's accelerated gradient descent is closely related to the ordinary
gradient descent algorithm.
The first step is identical, each following step carries some momentum of the
step before, thus stabilizing the procedure.
It is an optimal first-order optimization schema, i.e.~one that cannot be
improved asymptotically.
It archives quadratic convergence.
This property is retained when combined with the proximal-splitting procedure \cref{alg:ista}, the result is called \fista~\cite{fista}.
Each step of \fista\ evaluates the gradient and the proximal operator once, just as \ista does.
This means that the accelerated algorithm has a comparable cost for each iteration.

\sidetitle{Linesearch}
Another problem with \cref{alg:ista} is its dependence on the Lipschitz constant of
\(\bm{\nabla} f\) to determine the optimal stepsize.
For our choice of \(f\), the best constant \(L\) is given by~\cref{eq:lipf}.
To avoid this expensive calculation, we use a backtracking line search to
determine a suitable stepsize.
In this line search procedure we use \cref{eq:goal-approx} as an upper bound for \cref{eq:composite-goal}.
We do this by iterating and finding the smallest \(L\) for which
\cref{eq:goal-approx} is an upper bound.
This always results in the Lipschitz constant~\cite{fista}.
It is then straightforward to derive \cref{alg:linesearch}.

\begin{algorithm}[h!]
 \caption{Linesearch~\cite{fista}}\label{alg:linesearch}
 \begin{algorithmic}[1]
  \Require{\(L > 0, \eta > 1, \bm{\alpha}\)} 
  \Statex
  \Function{Linesearch}{$\bm{\alpha}, L$}
    \Let{\(i\)}{\(0\)}
    \Do
      \Let{L}{\(\eta^i L\)}
      \Let{prox}{\(\pi_{\bm{\alpha}} (\bm{\alpha}, L)\)}
      \Let{\(i\)}{\(i + 1\)}
    \doWhile{$F(\text{prox}) < Q_L(\text{prox}, \bm{\alpha})$}
    \State \Return{prox and \(L\)} \Comment{Also return prox to avoid duplicate calculations.}
  \EndFunction
 \end{algorithmic}
\end{algorithm}

We need to evaluate the line search once for each iteration step.
It is possible that this procedure finds a non-optimal \(L\), i.e.~an \(L\) that
is larger than the Lipschitz constant.
This leads to a smaller stepsize, which is not a problem in practice, because
our optimization procedure still converges, although slower than possible.
We have to take that into consideration for our choice of linesearch parameters.
Usual values are \(L = 0.5\) and~\(\eta = 2\).
Using \cref{alg:linesearch} we can finally present an optimal first order optimization algorithm, shown by \cref{alg:fista}.

It is of course possible to use a constant stepsize like in \cref{alg:ista}.
To do this, replace the line search with the minimal value of \(L\) and
calculate \(\pi(\bm{\alpha}, L)\) directly.
We can also integrate the linesearch into the \ista algorithm, by replacing the
fixed \(L\) with a call to the linesearch subroutine.
A comparison of the practical speed of \ista and \fista with constant stepsize can be seen in~\cref{fig:fista-convergence}.

An alternative backtracking scheme for \fista is offered in~\cite{fista-backtracking}.
It offers the same asymptotic convergence speed, but shows practical improvements for some minimization problems.

\begin{figure}[bt]
  \includegraphics[width=\textwidth]{fista}
  \caption[Comparison of \fista and \ista]{Comparison of \fista and \ista, both with constant stepsize.
    The figure shows the training \textsc{mse} for 500
    iterations with \(\lambda = 0.1\) for the training set of the concrete
    data set.
    The dotted lines indicate the best value reached by \ista.
    Note that \fista is able to return a better result after only 56
    iterations ---
    compared to the 500 needed by \ista.
  }\label{fig:fista-convergence}
\end{figure}

\begin{algorithm}[h]
 \caption{Fast Iterative Shrinkage Tresholding Algorithm (\fista)~\cite{fista}}\label{alg:fista} 
 \begin{algorithmic}[1]
   \Require{Initial guess for Lipschitz constant \(L\) of \(\bm{\nabla} f\), regularization parameter \(\lambda\)}
    \Statex
    \Function{Fista}{$L, \bm{\alpha}$} \Comment{\(\bm{\alpha}\) is an initial
      guess for \(\alpha^*\).}
      \Let{\(\bm{y}\)}{\(\bm{\alpha}\)}
      \Let{\(t\)}{\(1\)}
      \While{not converged}
        \Let{\(\bm{\alpha}_{\text{before}} \)}{\(\bm{\alpha}\)}
        \Let{$\bm{\alpha}, L$}{\Call{Linesearch}{$ \bm{y}, L$ }} \Comment{Linesearch returns \(\pi_L (\bm{y})\) and the used L.}
        \Let{\(t_\text{before}\)}{\(t\)}
        \Let{\(t\)}{\(\nicefrac{1}{2} (1 + \sqrt{1+4t^2}) \)}
        \Let{\(\bm{y}\)}{\(\bm{\alpha} + \left( t_\text{before} -1 \right) t^{-1} 
                    \left(\bm{\alpha} - \bm{\alpha}_{\text{before}} \right)\)}
      \EndWhile
     \State \Return{\(\bm{\alpha}\)}
    \EndFunction
\end{algorithmic}
\end{algorithm}

%%% Local Variables:
%%% mode: latex
%%% TeX-master: "../main"
%%% End:

\FloatBarrier{}
\subsection{Implementation}

\tikzumlset{fill class = white, fill template = white, fill package = white}
% \newcommand{\fistalb}{\\\phantom{+ solve(}}
% \begin{figure}[hbt]
% \begin{tikzpicture}
%   \begin{umlpackage}[x=0, y=0]{Fista}
%     \umlclass[type=abstact, width=20ex]{sgpp::solver::FistaBase}{
      
%     }{
%         % + solve(op : OperationMultipleEval, \\\phantom{+ solve} weights : DataVector,
%         % y : DataVector,\\\phantom{+ solve} maxIt : size\_t, threshold : double) : void
%       \umlvirt{\parbox[c]{0.7\textwidth}{+ solve(op : OperationMultipleEval,
%           \fistalb weights : DataVector,
%         y : DataVector, maxIt : size\_t, \fistalb threshold : double, L : double) : void}}

%     }

%     \umlclass[template={F}, y=-4, name=Fista]{sgpp::solver::Fista}{
%     - regularizationFunction : F 
%     }{
%       + Fista( regularizationFunction : F)\\
%     }
       
%     \umlinherit[]{sgpp::solver::FistaBase}{sgpp::solver::Fista}
%   \end{umlpackage}
% \end{tikzpicture}
% \caption{\textsc{Uml}-diagram of the implementation of \fista.}
% \end{figure}


\begin{figure}[h]
  %https://tex.stackexchange.com/questions/140739/extending-figures-into-the-margin-on-even-vs-odd-pages
  \centering
 \advance\hsize\marginparwidth
 \advance\hsize\marginparsep
 \begin{tikzpicture}
    \begin{umlpackage}{RegularizationFunctions}
      \umlclass[type=abstact, x=1]{RegularizationFunction}{
      }{
        \umlvirt{+ eval(weights : DataVector) : double}\\
        \umlvirt{+ prox(weights : DataVector, stepsize : double) : DataVector}
      }

      \umlclass[y=0, x=10]{ZeroFunction}{}{
        + ZeroFunction()
      }

      \umlclass[y=6, x=0]{RidgeFunction}{
        - lambda : double
      }{
       + RidgeFunction(lambda : double)}

      \umlclass[y=-4, x=0]{LassoFunction}{
        - lambda : double
      }{
      + LassoFunction(lambda : double)}

      \umlclass[y=-4, x=8]{ElasticNetFunction}{
        - lassoFunc : LassoFunction \\
        - lambda1 : double\\
        - lambda2 : double
      }{
        + ElasticNetFunction(lambda : double, l1Ratio : double)
      }

      \umlclass[y=5, x=8]{GroupLassoFunction}{
        - lambda : double\\
        - groups : map<coords, int>\\
        - storage : gridStorage*\\
        - groupMap : map<int, index>\\
        - lastSize: int
      }{
        + GroupLassoFunction(lambda : double,\\
        \phantom{+ } storage : GridStorage*)\\
        - calculateNorm(weights : DataVector) : DataVector\\
        - calculateIndices(weights : DataVector) : void
      }

    \umlinherit[]{ZeroFunction}{RegularizationFunction} 
    \umlinherit[]{RidgeFunction}{RegularizationFunction} 
    \umlinherit[]{LassoFunction}{RegularizationFunction} 
    \umlinherit[]{ElasticNetFunction}{RegularizationFunction} 
    \umlinherit[]{GroupLassoFunction}{RegularizationFunction} 

    \umlassoc{LassoFunction}{ElasticNetFunction}
    \end{umlpackage}
  \end{tikzpicture}
  \caption{\textsc{Uml}-class diagram of the implementation of the
    regularization functions.}
\label{fig:uml-regularization-functions}
\end{figure}

As seen in \cref{fig:uml-regularization-functions} we define a base class RegularizationFunction that offers the two methods \textsc{eval} and \textsc{prox}, which calculate \(g(\bm{\alpha})\) and its proximal operator respectively.
Every parameter that is needed for each functional is passed during construction, this way we achieve a great amount of flexibility.
Classes that use the regularization functions do not have to be concerned about the definition or the parameters of the functionals, they can treat them as a black box.
We offer an implementation of the ridge, the lasso, the elastic net and the group lasso function, which resemble the definitions outlined in this chapter.
The group lasso function uses the following subroutines:
\begin{description}
\item[calculateIndices] that partitions our weights into groups that share the
  same order. This method is only called when the grid size changes, so only
  once during the first solver iteration. When the grid size changes, e.g.~due
  to an adaptivity process, the groups are automatically recalculated.
  The results of this operation are stored in the vector \emph{groups} and the map
  \emph{groupMap}.
\item[calculateNorm] is called once per evaluation of \textsc{prox} and
  \textsc{eval}. It calculates the group norms and the group size.
\end{description}
The ElasticNetFunction uses the \emph{LassoFunction} to calculate the  \(l_1\) part of its evaluation.
Additionally we implemented a \emph{ZeroFunction} that can be used, when no regularization is desired.

Our solver \fista is then implemented as a class with one template argument: the proximal operator.
We therefore have to first create a \emph{RegularizationFunction} and then a \emph{Fista} class.
Even though this might seem inconvenient, it allows the compiler to inline all
calls to both the function evaluation and the proximal operator, which get
called once per iteration.
We defined a base class \emph{FistaBase} that allows us to hold a pointer not only to a specialized Fista object, but also to one where we do not know the used \emph{RegularizationFunction}.

\fista itself is split into various subroutines, that allow a separation of concerns.
This leads to a very clean implementation that closely represents
\cref{alg:fista} in combination with \cref{alg:linesearch}.
All calls to the subroutines were inlined automatically, at least for the used \textsc{gcc}-compiler with the highest optimization settings.
The \emph{Fista} class has two public methods.
We use the first one, called \textsc{solve}, to solve a specific problem instance
and the second one, called \textsc{getL}, to get the last estimated Lipschitz constant.
This is useful to avoid a costly grid search after refining the grid, because the Lipschitz constant is usually larger for a larger grid.

\FloatBarrier{}
\subsection{Results \textit{\&} Discussion}
We begin this section by analyzing the implicit feature selection performed by
all discussed methods.
To do this, it is helpful to use an artificial dataset.
\sidetitle{Friedman1 Dataset}
In our case we use the Friedman1 dataset, first published in~\cite{datasets-friedman}.
Let \(\bm{x} = (x_1, \ldots, x_{10}) \in \mathbb{R}^{10}\) be a uniformly distributed vector.
We use \(\bm{x}\) as our predictors, and define
\begin{equation}\label{eq:friedman1}
  y = 10 \sin(\pi x_1 x_2) + 20(x_3 - 0.5)^2 + 10x_4 + 5x_5 + \varepsilon,
\end{equation}
with \(\varepsilon \sim \mathcal{N}(0,1)\) as additional Gaussian noise.
Despite its simplicity, this dataset is very useful for evaluating the
regularization methods discussed in this section.
Its advantages are:
\begin{itemize}
\item The dataset is inherently sparse because five features are not correlated with the
  response and are left entirely unused.
\item It is completely additive with the exception of the \(x_1 \times x_2\)
  interaction term.
\item The importance of each variable is directly visible from the definition,
  which allows us to discuss the selection order.
  We can see that the contribution of the \(x_4\) term is the largest, followed
  by \(x_5, x_3\), and finally \(x_1, x_2\) and their interaction.
\item We know that the optimal \textsc{rmse} is equal to the standard deviation
  of the noise and is hence 1.0.
  The dataset can therefore be used as a sanity check.
\end{itemize}

Using a method discussed in~\cite{regularizationpaths}, we can calculate the value of \(\lambda\) for which all weights are exactly zero.
The maximum \(\lambda\) is given by
\begin{equation*}
  \lambda_{\text{max}} = \frac{\max_i \vert \langle \bm{\Phi}_i, \bm{y} \rangle \vert}{\lambda_2 n},
\end{equation*}
where the index \(i\) denotes the \(i\)th column of the matrix and \(\lambda_2\) is the amount of \(l_1\) regularization.
For the group lasso penalty we have to consider the group structure as well.
In practice, the bias term is often the group with the largest weights, in which
case we can use the same formula.
This is true for the Friedman1 dataset.
We then construct a logarithmic grid from \(\lambda_{\text{max}}\) to \(\lambda_{\text{min}} = \varepsilon \lambda_{\text{max}}\), where \(\varepsilon\) is set to 0.001.
It is more efficient to start the path with all weights set to zero,
see~\cite{regularizationpaths} for a more advanced discussion.

We calculated regularization paths for the lasso, the elastic net (with
\(\lambda_2 = 0.3\)) and the group lasso using an estimator with level 2 for the
Friedman1 dataset.

\begin{figure}[hbt]
  \makebox[\textwidth][l]{\includegraphics[width=1.2\textwidth]{path_lasso}}%
  \caption{Regularization path for the lasso}\label{fig:path-lasso}
\end{figure}
\sidetitle{Lasso path}
\Cref{fig:path-lasso} shows the path for the lasso, which first integrated the bias, then one \(x_4\) point, then
one each of \(x_1, x_2\), and one \(x_5\) point.
The only unneeded term included at \(\lambda_\text{max}\) was a \(x_8\) point with a very small weight.
This agrees roughly with the importance of the terms.
Note that it did not select all grid points of a group at the same time.
This has the effect that the inclusion of the second term of the same group often leads to a weight decrease of the first included term.
In other words the magnitude of a weight might decrease for a smaller regularization parameter.

\begin{figure}[hbt]
  \makebox[\textwidth][l]{\includegraphics[width=1.2\textwidth]{path_grp}}%
  \caption{Regularization path for the group lasso}\label{fig:path-grp}
\end{figure}

\sidetitle{Group Lasso Path}
The regularization path of the group lasso is depicted by \cref{fig:path-grp}.
It performed a more stable grid point selection and all grid points
corresponding to the same group were either in the model or not.
There was no sparsity on a grid point level but rather on the group level.
As with the lasso, the \(x_4\) terms were chosen first, followed by the \(x_1,
x_2\) terms at roughly the same time, the final chosen terms were the \(x_5\) and
\(x_3\) ones.
It did not choose any unneeded point, even at the minimum \(\lambda\).
Interestingly, the weights at \(\lambda_\text{max}\) for both the grouped and
standard lasso were very similar.

\begin{figure}[hbt]
  \makebox[\textwidth][l]{\includegraphics[width=1.2\textwidth]{path_en}}%
  \caption{Regularization path for the elastic net with \(\lambda_2 = 0.3\)}\label{fig:path-en}
\end{figure}

\sidetitle{Elastic Net Path}
The elastic net path, shown by \cref{fig:path-en}, also selected the bias first.
It then selected one point of \(x_1, x_2, x_4,\text{and } x_5\) each.
One \(x_3, x_6, x_7, x_8, x_9 \text{ and } x_{10}\) terms were selected next,
which are all irrelevant features, with the exception of the \(x_3\) point.
For the highest chosen \(\lambda\) all grid points were selected.
We can see that the elastic net does not perform proper feature selection when the ridge regularization dominates the lasso penalty.
The same reasons lead to the effect that the value of most weights were larger than in the lasso estimate, which agrees with the weight prior.
The result would look drastically different if a value of \(\lambda_2\) close to one is chosen, which then approaches the lasso penalty.

Using the same method we calculated the regularization paths for the lasso and
the grouped lasso again for the Friedman1 dataset, but this time using a level
three estimator.
The results are depicted in \vref{fig:friedman1-heatmap}.
We can see that while the group norms were similar, the grid point selection of
the group lasso was more stable and selected points on a group level.

\begin{figure}[p]
  \thisfloatpagestyle{empty}
  \begin{subfigure}[t]{0.5\textwidth}
    \centering
    \includegraphics[width=0.95\textheight, angle =-90 ]{heatmap_lasso}
    \caption{Lasso}
  \end{subfigure}\qquad
  \begin{subfigure}[t]{0.5\textwidth}
    \centering
    \includegraphics[width=0.95\textheight, angle =-90]{heatmap_grp}
    \caption{Group Lasso}
  \end{subfigure}
  \caption{Regularization paths for the Friedman1 dataset with level three.
  Each figure shows two heatmaps.
  The first one shows the \(l_2\) norm of the group terms, the second one presents the number of selected terms.}\label{fig:friedman1-heatmap}
\end{figure}

\sidetitle{Concrete Results}
We saw some results that indicated that the feature selection works.
The Friedman1 dataset is not a good choice to discuss the real-world performance
of our proposed penalties, because it is too simplistic.
This is why we return to the concrete dataset.
Results for the this dataset can be found in \cref{fig:sparse-results}.
We performed a grid search over \(\lambda \in [10^{0}, 10^{-1}, \ldots,
10^{-4}]\) again using five refinement steps over three points each.

For the level four grid, the group lasso performed best, followed by the elastic
net with \(\lambda_2 = 0.95\) and the lasso penalty.
We can see that we did not profit from a larger amount of ridge shrinkage, even
though a small ridge portion helped to stabilize the result.
The results indicate that sparsity on a group level was better than sparsity on a
basis-function level.
Note that the testing-\textsc{rmse} was best for the lasso.
We have seen that for a medium-sized grid all sparsity-inducing penalties showed
solid results.

Does this situation change for a larger grid?
The results for the level five grid showed the same tendency.
We can see that all methods achieved better results for the larger grid, only
the order of the methods is changed.
For this experiment the lasso showed the best results, followed by the elastic net with
\(\lambda_2 = 0.95\) and the group lasso.
This makes intuitive sense:
A larger grid needs more sparsity and sparsity on the group level is not enough
any more.
The result for the elastic net is interesting, because we had more grid points
than data points.
We would expect it to outperform the \(l_1\)-regularization, because it should show a more stable feature selection in this case.
All sparsity-inducing penalties showed promising results for the larger grid as well.

\begin{table}[tb]
  \centering
   \begin{tabular}[c]{c
    c
    S[table-format=1.0, table-figures-exponent=2, table-sign-mantissa, table-sign-exponent]
    %S[table-format=4.1(4)]
    S[table-format=1.4]
    S[table-format=4.0]
    S[table-format=2.3]
    S[table-format=2.3]}
  \toprule \multicolumn{1}{c}{Level}
& \multicolumn{1}{c}{Reg.~Method}
& \multicolumn{1}{c}{\(\lambda\)}
%& \multicolumn{1}{c}{\textsc{cv}-Grid}
& \multicolumn{1}{c}{\textsc{cv-rmse}}
& \multicolumn{1}{c}{Train-Grid}
& \multicolumn{1}{c}{Train-\textsc{rmse}}
& \multicolumn{1}{r}{Test-\textsc{rmse}}\\
     \midrule
4 & Group Lasso & 1e-02 & 4.774 & 1385 & 3.133 & 4.032\\
4 & Elastic Net ($\lambda_2$ = 0.95) & 1e-02 & 4.868 & 1382 & 2.902 & 3.869\\
4 & Lasso & 1e-02 & 4.884 & 1382 & 2.911 & 3.850\\
4 & Elastic Net ($\lambda_2$ = 0.5) & 1e-02 & 4.927 & 1384 & 3.077 & 4.046\\
4 & Ridge & 2e-02 & 5.007 & 1470 & 3.120 & 4.198\\
4 & Elastic Net ($\lambda_2$ = 0.05) & 1e-03 & 5.060 & 1367 & 2.483 & 3.813\\
\midrule
5 & Lasso & 1e-02 & 4.582 & 6632 & 2.470 & 3.737\\
5 & Elastic Net ($\lambda_2$ = 0.95) & 1e-02 & 4.594 & 6632 & 2.460 & 3.742\\
5 & Group Lasso & 1e-02 & 4.650 & 6694 & 2.861 & 3.955\\
5 & Elastic Net ($\lambda_2$ = 0.5) & 1e-02 & 4.669 & 6633 & 2.400 & 3.844\\
5 & Ridge & 2e-02 & 4.709 & 6650 & 2.286 & 4.184\\
5 & Elastic Net ($\lambda_2$ = 0.05) & 1e-02 & 4.730 & 6648 & 2.297 & 4.085\\
\bottomrule
   \end{tabular} 
   \caption{Results for the concrete dataset using different penalties.
   Entries are ordered by cross validation error.}\label{fig:sparse-results}
\end{table}

\sidetitle{Akaike information criterion}
Instead of comparing the average cross-validation error, we can also use the Akaike information criterion (\textsc{Aic})~\cite{aic}.
It is another way of comparing the relative quality of various models.
We can calculate it directly from the \textsc{mse}
\begin{align}
  \operatorname{\textsc{Aic}}(\textsc{df}, \textsc{mse}) = 2\operatorname{df} + n \ln{(\textsc{mse})} + c,
\end{align}
where \textsc{df} corresponds to the degrees of freedom of the model and \(n\) to the number of data points.
The constant \(c\)  depends only on the data and can be safely omitted,
because we only compare models for a specific dataset~\cite{esl}.
Note that the \textsc{Aic} is asymptotically equivalent to leave-one out
cross-validation~\cite{aic-asymp}.

\sidetitle{Effective Degrees of Freedom}
We use unbiased estimates of the true effective \textsc{df}s for the identity matrix and
the lasso regularization.
We start with the surprisingly simple solution for the lasso penalty
\begin{align*}
  \operatorname{df}(\bm{\Phi, \alpha}) &= \operatorname{rank}(\bm{\Phi}_{\bm{\mathcal{A}(\alpha)}}) \intertext{with}
  \bm{\mathcal{A}}(\bm{\alpha}) &= \{a \in \bm{\alpha}\,|\,a \neq 0\},
\end{align*}
where \(\bm{\Phi}_{\bm{\mathcal{A}(\alpha)}}\) denotes all columns of
  \(\bm{\Phi}\) that are in the so-called active set \(\bm{\mathcal{A}}\) and
  \(\operatorname{rank}(\bm{\Phi})\) corresponds to the rank of the model matrix \(\bm{\Phi}\)~\cite{lasso-df}.
If the model matrix \(\bm{\Phi}\) has full column rank the \textsc{df}s are given by the cardinality of
the active set~\cite{lasso-df}.
The solution for the identity matrix regularization is given by
\begin{equation*}
 \operatorname{df}(\bm{\Phi}, \lambda) = \sum_i \frac{\sigma_i^2 - \lambda}{\sigma_i^2},
\end{equation*}
where \(\sigma_i\) represents the ith singular value of \(\bm{\Phi}\)~\cite{esl}.
For the sake of brevity, the formulas for our other used penalties are omitted.
Unbiased estimates for the elastic net can be found in~\cite{lasso-df} and for the group
lasso in~\cite{grouplasso-df}.
The formula for the diagonal Tikhonov regularization can be derived easily from
the general formula presented for example in~\cite{esl}.

\begin{table}[h]
  \centering
   \begin{tabular}[c]{S[table-format=1]
    c
    S[table-format=1.2, table-figures-exponent=2, table-sign-mantissa, table-sign-exponent]
    S[table-format=4]
    S[table-format=3.1] %df
    S[table-format=2.3] %rmse - train
    S[table-format=4.2] %aic 
    S[table-format=2.3]}%rmse - test
  \toprule \multicolumn{1}{c}{Level}
& \multicolumn{1}{c}{Reg.~Method}
& \multicolumn{1}{c}{\(\lambda\)}
& \multicolumn{1}{c}{Train-Grid}
& \multicolumn{1}{c}{\textsc{df}}
& \multicolumn{1}{c}{Train-\textsc{rmse}}
& \multicolumn{1}{c}{\textsc{Aic}}
& \multicolumn{1}{r}{Test-\textsc{rmse}}\\
     \midrule
5 & Ridge & 1.96e-02 & 6650 & 716.7 & 2.286 & 2796.22 & 4.184 \\
4 & Lasso & 1.00e-02 & 1382 & 518.0 & 2.911 & 2797.09 & 3.850 \\
4 & Ridge & 1.84e-02 & 1470 & 558.0 & 3.120 & 2991.22 & 4.198 \\
5 & Lasso & 1.00e-02 & 6632 & 754.0 & 2.470 & 2997.94 & 3.737 \\
\bottomrule
   \end{tabular} 
   \caption{Comparison of \textsc{Aic} for the concrete dataset.
     All results are ordered by \textsc{Aic} in increasing order.}\label{fig:sparse-aic}
\end{table}

\Cref{fig:sparse-aic} shows the \textsc{Aic} for the best found ridge and lasso models for both level four and five for the concrete data set.
All estimators used five refinements on three points each.
We can see that the number of degrees of freedom did not increase linearly with
larger grid size.
This is because we only used about one thousand training data points, which was the limiting factor, rather than the number of grid points.
While the ridge penalty used more degrees of freedom for the grid with level
four than the lasso, the situation was reversed for the level five grid.
The \textsc{Aic} for both methods was comparable, it was smaller for both ridge
models by a very small margin.

We can see that all regularization methods managed to perform well for a
large grid that has roughly six times more grid points than data points.
The \textsc{Cv}-errors achieved were better than the best results for both tested
Tikhonov regularization methods.

%%% Local Variables:
%%% mode: latex
%%% TeX-master: "../main"
%%% End:

