\thispagestyle{empty}

% \vspace*{20mm}

% \begin{center}
% %{\usekomafont{section} Acknowledgments}
% \end{center}

% \vspace{10mm}

\renewcommand{\dictumwidth}{\textwidth}
\setchapterpreamble{\dictum[Walter Moers - Rumo and His Miraculous Adventures]{
  ‘Yes, it’s pretty dark in here,’ said the professor. ‘But there are miracles that can only occur in the dark.’\\
‘Where’s the exit, please?’\\
‘You don’t want to try out my oracle?’\\
‘Er, to be honest: no. I’m not feeling too good and I’ve had it up to here with this fairground hocus-pocus.’\\
Nightingale’s eyes flashed and something inside his head crackled ominously. ‘Hocus-pocus?’ he hissed. ‘This isn’t hocus-pocus, it’s scientific exactitude!’
}}
%\addcontentsline{toc}{chapter}{Acknowledgments}
\chapter{Acknowledgments}
\vspace{1em}
Darkness is the natural space for even small scientific works.
But next to darkness, just around the corner, lingers the light of social interaction.

I want to thank the department of scientific computing for the knowledge they
kindly gave me during the two seminars and the practical course I had the honour
to take part in.

I would also like to thank Prof. Bungartz for the chance to work on this exciting topic.
Special thanks goes to my advisor Valeriy, who managed to steer me in the right direction and always had time to answer any of my questions.
His valuable feedback not only improved the quality of this thesis, but also
improved my knowledge of scientific writing.

My gratitude extends to my friends and family, who always managed to motivate me.
A honorary mention goes to my friend William, who did the (mostly) thankless task of proof-reading this thesis.
Of course, only I am to blame for all errors left.

Without any of the mentioned people, this thesis might have been impossible.
In any case, it would have been a far less enjoyable experience.

\cleardoublepage{}

%%% Local Variables:
%%% mode: latex
%%% TeX-master: "../main"
%%% End:
